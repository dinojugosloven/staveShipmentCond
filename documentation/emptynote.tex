%-------------------------------------------------------------------------------
% This file provides a skeleton ATLAS document
%-------------------------------------------------------------------------------
\documentclass{atlasnote} 
% Options:
%		nomaketitle Do not run \maketitle from the class
%		coverpage	  Create ATLAS draft cover page for collaboration circulation
%							  atlascover.sty must be in this directory or a system directory.
%							  See coveronly.tex for a list of variables that should be defined.
%							  Do not include the hyperref package in this file when you set coverpage.
%		CONF			  This is a CONF note
%		usetikz		  Load tikz package in the right place

%-------------------------------------------------------------------------------
% Extra packages:
% See doc/atlasphyics.pdf for a list of the defined symbols
%\usepackage{atlasphysics}
% Comment out hyperref if you use coverpage
%\usepackage[colorlinks,breaklinks,pdftitle={ATLAS draft},pdfauthor={The ATLAS Collaboration}]{hyperref}  
%\hypersetup{linkcolor=blue,citecolor=blue,filecolor=black,urlcolor=blue} 

%-------------------------------------------------------------------------------
% Specify title, author, abstract and document numbers here

% Title
\title{A template for ATLAS documents}

% Author --  Default is ``The ATLAS collaboration''
%\author{The ATLAS Collaboration}

% if multiple authors/affiliations are needed, use the authblk package
\usepackage{authblk}
\renewcommand\Authands{, } % avoid ``. and'' for last author
\renewcommand\Affilfont{\itshape\small} % affiliation formatting

\author[a]{First Author}
\author[a]{Second Author}
\author[b]{Third Author}

\affil[a]{One Institution}
\affil[b]{Another Institution}

% Date: if not given, uses current date
%\date{\today}

% Draft version: if given, adds draft version on front page, a
% 'DRAFT' box on top of each other page, and line numbers to easy
% commenting. Comment or remove in final version.
\draftversion{x.y}

% Journal: adds a 
\journal{Phys. Lett. B} 

% Abstract
\abstracttext{
  This is the abstract
}

%-------------------------------------------------------------------------------
% Content
%-------------------------------------------------------------------------------
\begin{document}

%\maketitle

% List of contributors to the analysis
% The width of the name is an optional argument
\begin{atlascontribute}
\item[Student, Joe] fake background estimate in muon channel, MCFM
  calculations of diboson background cross section limit calculation
\item[Student, Jane] top-quark background estimate, final
  analysis implementation
\item[Postdoc, John] editor of internal note, fake background
  estimate in the electron channel, cut optimisation studies
\item[Postdoc, Jack] major contributions to nearly all aspects of
  analysis, paper contact editor
\item[Professor, Joan] supervised student and postdoc of Institute 1,
  contributions to top-quark background estimate.
\item[Researcher, Joanne] implementation of muon calibration and systematics
\end{atlascontribute}
\clearpage

% Test of tikz.sty loading in atlasnote.cls. Use 'usetikz' option in documentclass declaration
%\input{tikz-test}

%-------------------------------------------------------------------------------
\section{Introduction}
\label{sec:intro}
%-------------------------------------------------------------------------------

Place your introduction here

%-------------------------------------------------------------------------------
\section{Results}
\label{sec:result}
%-------------------------------------------------------------------------------

Place your results here

%-------------------------------------------------------------------------------
\section{Conclusion}
\label{sec:conclusion}
%-------------------------------------------------------------------------------

Place your conclusion here

\end{document}